\section{Background}
This section provides an overview of all of the technologies mentioned in the remainder of the paper.
\subsection{HTTP}
HTTP is the most important protocol in this project and as such needs to be covered in detail.
\texttt{HTTP/1.1} is defined in \texttt{RFC2616}~\cite{rfc2616} and all of the information about HTTP originates from that document unless otherwise specified.
HTTP is a Request/Response protocol, which means the client requests data from the server, and the server returns it.\\
An example request is as follows:
\begin{verbatim}
GET / HTTP/1.1
Host: localhost
User-Agent: curl/7.58.0
Accept: */*
\end{verbatim}
In the above request, the request line comes first (\texttt{GET / HTTP/1.1}), followed by the headers. The headers are a set of key-value pairs which form a dictionary that tells the server extra information about the client.

One of the headers is:\\
\texttt{User-Agent: curl/7.58.0}\\
Which is the client telling the server that the \texttt{User-Agent} of the program used to get the page is \texttt{curl} version \texttt{7.58.0}.\vspace{0.4cm}\newline
An example response is as follows:
\begin{verbatim}
HTTP/1.0 200 OK
Server: SimpleHTTP/0.6 Python/3.6.4
Date: Tue, 20 Mar 2018 14:25:01 GMT
Content-type: text/html
Content-Length: 10088
Last-Modified: Tue, 13 Mar 2018 15:27:09 GMT

<!doctype html>
<html>
<head>
...
\end{verbatim}
In this response, the response line (the first line) is sent, along with some headers which provide important meta-data about the response itself, which is then included when the headers are finished.

\subsubsection{HTTP Encoding types}
HTTP has a variety of different encoding types and methods.
These are specified in the \texttt{Transfer-Encoding} and \texttt{Content-Encoding} headers, both of which are  used to perform compression and other encoding of data.\par
\texttt{Transfer-Encoding} additionally allows for \texttt{Chunked} encoding~\cite{rfc7230}, which splits the page or portions of data up into multiple sections, and sends the length for each section and the sections themselves separately. This is done that so web servers can send data as it is generated, rather than having to buffer the entire page. Chunked Encoding is built into HTTP/1.1, so that the client cannot request that the server does not use it.

\subsection{TCP}
TCP (Transmission Control Protocol) is defined in \texttt{RFC793}~\cite{rfc793} and all of the information about TCP originates from there unless otherwise specified.
TCP is a connection-based protocol that allows for reliable data transfer between a server and a client.
TCP connections are a connection between two sockets, one active and one passive.
A passive socket is listening, and is typically a \texttt{server} whereas an active socket is connecting and is typically a \texttt{client}.
The passive, or listening socket is opened by a server and waits for a connection from an active, or client socket.

\subsection{Steganography}
Steganography is defined as~\cite{dictsteno}:
\begin{quotation}
    `the art or practice of concealing a message, image, or file within
    another message, image, or file'
\end{quotation}
An example of this could be encoding data into an image, or into an HTTP stream.

\subsection{Obfuscation}
To obfuscate is defined as~\cite{dictobfs}:
\begin{quotation}
     `to be evasive, unclear, or confusing'
\end{quotation}
When applied to computers, this is similar to steganography and the two terms are often used interchangeably, however, steganography is the art of concealing and disguising data, whereas obfuscation makes data difficult to read and understand. 

\subsection{Encryption}
\begin{quotation}
    `Encryption is the process of converting data to an unrecognisable or
    `encrypted' form. It is commonly used to protect sensitive
    information so that only authorized parties can view it~\cite{dictenc}.'
\end{quotation}
This definition means that if some data has been encrypted with a sensible encryption scheme, it will be very difficult for a third party to decrypt.
There are many types of encryption, AES and RSA are two major examples.\par
It is widely referred to as bad practice to `\texttt{roll your own crypto}'~\cite{memtocrypto}, which means that it is a bad idea to write a bespoke cryptographic algorithm, or to try and come up with your own cryptographic scheme.

\subsection{Privacy}
Privacy on the internet is often overlooked, however, it is vitally important~\cite{privacyrulez}. Almost everything that is done on the internet is tracked by multiple parties: Internet service providers (ISPs), the websites visited, DNS servers, third party servers and many more. These parties could all be gathering information in order to profile a person and build a model to predict their behavior, which can then be used for highly targeted advertising --- not always desirable for the end user. An extreme example of highly targeted advertisement was a case in 2012 where a retailer inadvertently told a father that his daughter was pregnant by sending her coupons for baby clothes and cribs~\cite{babyshower}.
Since this case was first reported, statistical models have become significantly more complex and intrusive. Google, who are the worlds biggest advertiser, recently allowed users to mark Adverts as `knowing too much'~\cite{googlearewatching}, which demonstrates that the problem exists, and is perhaps endemic in the advertising industry.
As this case shows, lack of privacy is scary, and tools that increase privacy can only be a good thing.

\subsection{Blocking of Services}
When you connect to Public Wi-Fi, the provider will often limit what content you have access to, typically by blocking ports or preventing access to certain websites, often known as a blacklist.\\
HTTP is run over port 80, which is often one of the only ports available.

\subsection{Data Tunnelling}
In the field of networking in Computer Science, networking is split into layers.
Tunnelling is when some of the layers are encapsulated and sent as a regular payload.
A real-world representation of data encapsulation could be putting a letter inside an envelope, addressing it, stamping it, and then putting that envelope inside a second larger envelope. The larger envelope can then be addressed, stamped and posted independently.
On receipt of the larger envelope, the encapsulated envelope can then be taken out and posted. This has the effect of concealing that two parties are communicating, and is a good analogy of how encapsulation on a computer network functions.\newpage
Below is a diagram explaining how tunnelling works:
\begin{center}
\begin{tikzpicture}[-latex ,auto ,node distance =.7 cm and 6cm ,on grid,
    semithick, red/.style={top color=white, bottom color = red!20, draw,processblue, text=blue, minimum width=4 cm}, state/.style={top color=white, bottom color = processblue!20, draw,processblue, text=blue, minimum width=4 cm}, line/.style={color=black,line width=0.04cm}]
\node[state] (E)[]{IP Layer};
\node[state] (F)[above=of E]{TCP Layer};
\node[red] (G)[above=1cm of F]{IP Layer};
\node[red] (H)[above=of G]{TCP Layer};
\node[state] (A)[left=of G]{IP Layer};
\node[state] (B)[above=of A]{TCP Layer};
\node[state] (C)[above=of B]{Payload};
\node[red] (I)[above=of H]{Payload};
\node[fit=(G)(I), draw, thick, inner sep=0.2cm, blue,minimum width=4cm](PC1){};
\node[label] (J)[above=1.5cm of PC1]{Encapsulated Traffic};
\node[label] (D)[left=of J]{Typical Traffic};
\node[label] (K)[right=4.5cm of PC1]{Encapsulated Payload};
\draw[line] (A.east) -- (G.west);
\draw[line] (B.east) -- (H.west);
\draw[line] (C.east) -- (I.west);
\end{tikzpicture}
\end{center}


\subsection{VPN}
A VPN or Virtual Private Network (defined in \texttt{RFC2764}~\cite{rfc2764}) is a piece of software, split into a server and a client, where the server and client (or clients) form a network which is both virtual and private.\\
A typical example of a VPN is OpenVPN, \texttt{https://openvpn.net/}.\\
For a network to be private, it needs to be encrypted so no one else can see the data, and for it to be virtual, it has to only exist inside computers, and not physically connected by cables.\par In practical terms this means that data is tunnelled from one computer to another, generally by use of \texttt{tun} or \texttt{tap} devices:\\
\texttt{IP} packets can be read from and written to a \texttt{tun} device, whereas \texttt{Ethernet} frames can be read from and written to a \texttt{tap} device.\par
Tun and Tap devices are virtual devices that exist on a computer, packets routed down them can be read from the device as though it is a file.
Whichever device type is used, the packets or frames are transmitted through an existing connection to a remote client or server.
