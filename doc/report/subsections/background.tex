\section{Background}
This section provides an overview of all of the technologies mentioned in the remainder of the project.
\subsection{HTTP}
HTTP (Hyper Text Transfer Protocol) is the most important protocol in this project and as such needs to be covered in detail.
\texttt{HTTP/1.1} is defined in \texttt{RFC2616}\cite{rfc2616} and all of the information about HTTP comes from that document unless otherwise specified.
HTTP is a Request/Response protocol, which means the client requests data from the server, and the server returns it.\\
An example request is as follows:
\begin{verbatim}
GET / HTTP/1.1
Host: localhost
User-Agent: curl/7.58.0
Accept: */*
\end{verbatim}
In this request, the request line comes first (\texttt{GET / HTTP/1.1}), and the headers. The headers are a set of key-value pairs which form a dictionary that tells the server extra information about the client.

One of the headers is:\\
\texttt{User-Agent: curl/7.58.0}\\
This is the client telling the server that \texttt{User-Agent} or the program used to get the page is \texttt{curl} version \texttt{7.58.0}.\vspace{0.4cm}\newline
An example response is as follows:
\begin{verbatim}
HTTP/1.0 200 OK
Server: SimpleHTTP/0.6 Python/3.6.4
Date: Tue, 20 Mar 2018 14:25:01 GMT
Content-type: text/html
Content-Length: 10088
Last-Modified: Tue, 13 Mar 2018 15:27:09 GMT

<!doctype html>
<html>
<head>
...
\end{verbatim}
In this request the response line is sent, along with some headers which provide important meta-data about the response itself, which is then included when the headers are finished.

\subsubsection{HTTP Encoding types}
HTTP has a variety of different encoding types and methods.
These are specified in the \texttt{Transfer-Encoding} and \texttt{Content-Encoding}, both of which are  used to perform compression and other encoding of data.\par
\texttt{Transfer-Encoding} additionally allows for \texttt{Chunked} encoding\cite{rfc7230}, which splits the page or portions of data up into multiple sections, and sends the length for each section separately. This is done so web servers can send data as it is generated, rather than having to buffer the entire page. It is built into HTTP/1.1, so the client cannot request that the server does not use it.

\subsection{TCP}
TCP (Transmission Control Protocol) is defined in \texttt{RFC793}\cite{rfc793} and all of the information about TCP comes from there unless otherwise specified.
TCP is a connection based protocol that allows for reliable data transfer between a server and a client.
TCP connections are a connection between two sockets, one active and one passive.
A passive socket is listening, and is typically a \texttt{server} whereas an active socket is connecting and is typically a \texttt{client}.
The passive, or listening socket is opened by a server and waits for a connection from an active, or client socket.

\subsection{Steganography}
Steganography is defined as\cite{dictsteno}:
\begin{quotation}
    the art or practice of concealing a message, image, or file within
    another message, image, or file
\end{quotation}
An example of this could be encoding data into an image, or into a HTTP stream.

\subsection{Obfuscation}
To obfuscate is defined as\cite{dictobfs}:
\begin{quotation}
     to be evasive, unclear, or confusing
\end{quotation}
Obfuscation is simply the act of being evasive, unclear, or confusing.\par
When applied to computers, this is similar to steganography and the two terms are often used interchangeably, however steganography is the art of concealing and disguising data, whereas obfuscation just makes data difficult to read and understand. 

\subsection{Encryption}
\begin{quotation}
    Encryption is the process of converting data to an unrecognizable or
    `encrypted' form. It is commonly used to protect sensitive
    information so that only authorized parties can view it\cite{dictenc}.
\end{quotation}
This means that if I have encrypted some data with a sensible encryption scheme, it will be very difficult for a third party to decrypt.
There are many types of encryption, AES and RSA are two major examples.\par
It's generally referred to as bad practice to `\texttt{roll your own crypto}'\cite{memtocrypto}, which simply means that it is a bad idea to write a bespoke cryptographic algorithm, or to try and come up with your own cryptographic scheme.

\subsection{Privacy}
Privacy on the internet is often overlooked, however, it is vitally important\cite{privacyrulez}. Almost everything that is done on the internet is tracked by multiple parties: Internet service providers (ISPs), the websites visited, DNS servers, third party servers and many more. These parties could all be gathering information in order to profile and build a model to predict behavior. This can then be used for highly targeted advertising which is not always desirable for the end user. An extreme example of this was a case in 2012 where a retailer inadvertently told a father his daughter was pregnant by sending her coupons for baby clothes and cribs.\cite{babyshower}
This case was first reported in early 2012, and since then statistical models have become significantly more complex and intrusive. This is demonstrated by Google recently allowing you to mark Adverts as `\texttt{knowing too much}'\cite{googlearewatching}.
As this case shows, lack of privacy is scary, and tools that increase privacy can only be a good thing. 

\subsection{Blocking of services}
When you connect to Public Wi-Fi, often the provider will limit what content you have access to, often by blocking `ports' or preventing a blacklist of websites.\\
HTTP is run over port 80, and this is often one of the only ports available. ISP's have also been known to block certain websites.

\subsection{Data Tunnelling}
In the field of networking in Computer Science, networking is split into layers.
Tunnelling is when some of the layers are encapsulated and sent as a regular payload.
A real world example of this would be putting a letter inside an envelope, addressing it, stamping it, and then putting that envelope inside a second larger envelope. The second larger envelope can then be addressed, stamped and posted independently.
On receipt of the larger envelope, the one inside can then be taken out and posted. This has the effect of concealing that two parties are communicating, and is a good analogy of how encapsulation on a computer network functions.

\subsection{VPN}
A VPN or Virtual Private Network which is defined in \texttt{RFC2764}\cite{rfc2764}, is a piece of software, split into a server and a client, where the server and the client (or clients) form a network which is both virtual and private.\\
For a network to be private, it needs to be encrypted so no one else can see the data, and for it to be virtual, it only has to exist inside computers, and not physically connected by cables.\par In practical terms this means that data is tunnelled from one computer to another, generally by use of \texttt{tun} or \texttt{tap} devices:\\
\texttt{IP} packets can be read from and written to a \texttt{tun} device.\\
\texttt{Ethernet} frames can be read from and written to a \texttt{tap} device.\par
Tun and Tap devices are virtual devices that on a computer, and packets routed down them can be read from the device as though it is a file.
Which ever device type is used, the packets/frames are transmitted through an existing connection to a remote client or server, which will also read and write from a corresponding device.\par
A typical example of a VPN is OpenVPN, \texttt{https://openvpn.net/}.


