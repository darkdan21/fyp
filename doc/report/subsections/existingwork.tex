\section{Existing Work}
In this section, related works are explored, and some literature surrounding steganography, HTTP, DNS tunnelling, and detecting tunnelled traffic is discussed.

There has been much research done regarding steganography to hide data inside networking protocols, and a large proportion of this work has been done on DNS, as it is often below the radar for firewalls and checking if data is being exfiltrated.
\subsection{Image Steganography}
Steganography is most commonly used for hiding data in unused or unimportant areas of data~\cite{exploringsteno}, the most common place to hide data is inside images. This is the case because not all of the data in an image is required for a human to understand it, and the human eye is very good at filtering out noise.
Basic image steganography is carried out by changing the least significant bit or bits in the data, whereas more advanced approaches to steganography can involve identifying redundant data in images~\cite{introsteno} which is better, because it is more difficult to detect using steganalysis, which is the study of detecting messages hidden using steganography.

Steganography that is hidden from computers and is hidden from people are quite different things, and can require quite different approaches, it is therefore a much more significant challenge to hide data from both.

\subsection{DNS Steganography}
It is possible to hide data in DNS requests, otherwise known as DNS Tunnelling, which is often used to get around firewalls and hide which websites are being accessed, as Greg Farnham discusses in his paper `Detecting DNS Tunnelling'~\cite{detectingdns}. The main benefit of this is that it often works when there is no direct access to the internet because DNS requests are often propagated through firewalls.
The paper mentioned previously highlights the point raised in the previous section, that it is very easy for a human to look at the data and see it is abnormal, but non-trivial for a computer.
The paper describes how the data is detected, and in doing so describes in depth how the data is encoded and tunnelled.
DNS tunnelling as described in the aforementioned paper has a few advantages and disadvantages.
The key advantage is that it can be used in locked-down networks, as DNS traffic is often let out, but the main disadvantage is that data transfer is very slow with large amounts of overhead.
DNS tunnelling is slow because there is a very limited amount of data that can be encapsulated inside DNS requests and responses, DNS requests are never hugely fast and the results can become cached.
\subsection{HTTP Protocol}
The HTTP 1.1 Protocol~\cite{rfc2616} is a protocol that describes how data is sent to and from a client, and it contains many areas where data could be included, such as:
Images,
HTML,
CSS,
Javascript,
Binary files,
and more, all of which can be used to hide data.\par
The HTTP headers could also be an area where data could be concealed, as they are whitespace insensitive, and they represent a dictionary which means that the order in which they are sent does not matter.

\subsection{Detecting Tunnelled Traffic}
There are a variety of ways to detect traffic being tunnelled, one of the most common is to perform entropy analysis~\cite{detectingdns}.
Entropy analysis is looking at the `\texttt{randomness}' of the data to determine whether there is compressed or encrypted data within a stream. This works because compressed and encrypted data both strive to appear random (for different reasons), which means that they will have higher entropy than English text.\par
Another way of performing the lookup is to do character frequency analysis~\cite{freqanal}.
Frequency analysis looks at the difference in frequency of letters in expected tokens or English words and in random data. Frequency analysis is similar to but not quite the same as entropy analysis but can be more effective.
This method of analysis is often more effective because English words or tokens have slightly different frequency characteristics to large passages of text, so determining whether something is likely to appear in a domain name is slightly different to determining whether it will appear in a block of text.

